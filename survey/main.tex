\documentclass{article}
\usepackage[utf8]{inputenc}

\title{Survey on Discrete Wavelet Transforms Using GPU}
\author{Samuel Li}
\date{April 2015}

\usepackage{natbib}
\usepackage{graphicx}
\usepackage{color}
\newcommand{\fix}[1]{\textcolor{red}{#1}} %Put words in Red

\begin{document}

\maketitle

\begin{abstract}
Discrete Wavelet Transform (DWT) is widely used in data reduction applications.
%
However, such transforms are relatively computationally intensive.
%
To accelerate the calculation of DWT, researchers have applied various parallel
computing techniques, based on the distributed memory architecture~\cite{
chadha2002scalable, woo1995parallel, uhl1996wavelet, nielsen1997scalable}
and the shared memory architecture~\cite{
lucka2000parallel, uhl2000optimization,kutil1999hardware}.
%
In the recent years, the many-core architecture, such as GPU acceleration 
cards, have emerged in the parallel computing field.
%
In this survey paper, we cover many topics regarding calculation of DWT on GPUs.
%
More specifically, we start the survey from technical discussion and evaluation
of implementing the DWT on GPUs~\cite{tenllado2008parallel, van2011accelerating,
garcia2005gpu}.
%
Then we present a few successful use cases of GPU in parallel DWT~\cite{
strengert2004hierarchical, strengert2006pyramid, wong2007discrete,
treib2012turbulence}.
%
Finally, to prepare for the exascale computing, we survey the use of 
heterogeneous architecture with GPUs and multi-core CPUs together,
as well as the GPU clusters~\cite{franco2009parallel, franco2010parallel,
strengert2005large, franco20122d, rossinelli2011multicore}.
%
\fix{It would be interesting to see how this
field has evolved over the last 20 years (the time span of your
references) relative to the parallel architecture changes, especially in
the GPU architectures.}
\end{abstract}

\section{Introduction}


\bibliographystyle{plain}
\bibliography{main}
\end{document}
