\documentclass{article}
\usepackage[utf8]{inputenc}

\title{Porting Wavelet Transforms onto Many-core Architectures ---
        Using the EAVL Framework}
\author{Samuel Li}
\date{May 2015}

\usepackage{natbib}
\usepackage{graphicx}
\usepackage{color}
\newcommand{\fix}[1]{\textcolor{red}{#1}} %Put words in Red

\begin{document}

\maketitle

\begin{abstract}
\begin{itemize}
\item the area and problems to be explored, 
\item the motivations for choosing this area, 
\item possible directions of investigation, 
\item proposed type of project, 
\item the expected results.
\end{itemize}
\end{abstract}

\section{Introduction}
Wavelet transform is a technique rooted from the signal processing 
community~\cite{daubechies1990wavelet, mallat1999wavelet},
and soon people discovered its capacity in data reduction.
%
One of the most prominent use of wavelet in data reduction has been
the JPEG2000 still image compression
standard~\cite{adams2001jpeg,usevitch2001tutorial}. 

In the scientific simulation and visualization field, researchers have
explored the use of discrete wavelet transforms (DWT) in two 
relative different directions, both aiming data reduction.
%
In the first direction, researchers use the DWT to provide data access
in a multi-resolution fashion, meaning that an approximation of the volume data
set is loaded at a lower resolution at first, and the region of interest
is then reconstructed with a higher resolution~\cite{mallat1989theory,
kanai1998digital, baldwin2003multi}.
%
In the second direction, the volume data is reconstructed at the 
original resolution, but at a lossy manner~\cite{bethel2012high,
norton2012vapor}.

In a typical use case, the DWT is normally applied on each dimension of a
volume data set (e.g. X, Y, and Z dimension). 
%
Moreover, people always apply multiple rounds of DWT to achieve better
data reduction in applications.
%
The overall DWT on a volume data set is then becoming a heavy computation
task in most systems, thus a faster computation of DWT is much desired. 

Many-core architectures have emerged recently as accelerators for a 
traditional computer system.
%
On the one hand, these accelerator cards have much more compute units 
than a traditional CPU (some NVidia GPUs have thousands of compute units).
%
On the other hand, the compute units on these accelerator cards are 
relatively simple compared to a CPU, making them not suitable for complex 
computational tasks.

DWT requires repetitive calculation on arrays of data, with little to no
dependency between different arrays.
%
Such problems would perfectly fit in the many-core architecture, once we
successfully implement the DWT calculation of arrays on a target many-core
architecture.
%
In this term project, I am going to explore porting the DWT onto the 
many-core architecture.


\section{Implementation Plan}
The implementation of porting the DWT onto the many-core architectures
is based on two parts of existing work: 1) the VAPOR software 
package~\cite{clyne2007interactive},
and 2) the EAVL framework~\cite{meredith2012distributed}.




\bibliographystyle{plain}
\bibliography{main}
\end{document}
