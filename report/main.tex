\documentclass{article}
\usepackage[utf8]{inputenc}

\title{Implementing Wavelet Transforms onto Many-core Architectures ---
        Using the EAVL Framework}
\author{Samuel Li}
\date{May 2015}

\usepackage{natbib}
\usepackage{graphicx}
\usepackage{color}
\newcommand{\fix}[1]{\textcolor{red}{#1}} %Put words in Red

\begin{document}

\maketitle

% \begin{abstract}
% \end{abstract}

\section{Introduction}
\label{sec:intro}
%
Wavelet transform is a technique rooted from the signal processing 
community~\cite{daubechies1990wavelet, mallat1999wavelet},
and soon people discovered its capacity in data reduction.
%
One of the most prominent uses of wavelet in data reduction has been
the JPEG2000 still image compression
standard~\cite{adams2001jpeg,usevitch2001tutorial}. 

In the scientific simulation and visualization field, researchers have
explored the use of discrete wavelet transforms (DWT) in two 
relative different directions, both aiming data reduction.
%
In the first direction, researchers use the DWT to provide data access
in a multi-resolution fashion, meaning that an approximation of the volume data
set is loaded at a lower resolution at first, and the particular region of interest
is then reconstructed with higher resolutions~\cite{mallat1989theory,
kanai1998digital, baldwin2003multi}.
%
In the second direction, the volume data is reconstructed at the 
original resolution, but using a subset of the wavelet transformed 
data.
%
The second direction results in a lossy compression~\cite{bethel2012high,
norton2012vapor}.

In a typical use case, the DWT is normally applied on each dimension of a
volume data set (e.g. X, Y, and Z dimension). 
%
Moreover, in practice, people always apply multiple rounds of DWT to achieve better
data reduction. 
%
The overall DWT on a volume data set is then becoming a heavy computation
task in most systems, thus a faster computation of DWT is much desired. 

Many-core architectures have emerged recently as accelerators for a 
traditional computer system.
%
On the one hand, these accelerator cards have much more compute units 
than a traditional CPU (some NVidia GPUs have thousands of compute units).
%
On the other hand, the compute units on these accelerator cards are 
relatively simple compared to a CPU, making them not suitable for complex 
computational tasks.

DWT requires repetitive calculation on arrays of data, with little to no
dependency between different arrays.
%
Such problems would perfectly fit into the many-core architecture, once we
successfully implemented the DWT calculation of a single array on a 
target many-core architecture.
%
In this term project, I am going to explore parallelizing the DWT algorithm 
to fit into the many-core architectures.


\section{Implementation Plan}
The implementation of porting the DWT onto the many-core architectures
is based on two existing open-sourced projects: 1) the VAPOR software 
package~\cite{clyne2007interactive},
and 2) the EAVL framework~\cite{meredith2012distributed}.

VAPOR is a scientific visualization package, and it highlights the 
support for data reduction in both directions we talked in
Section~\ref{sec:intro}: multi-resolution and lossy compression.
%
VAPOR's data reduction functionality is achieved by using wavelet transforms.
%
Moreover, VAPOR is able to use three different forms of wavelet, which 
are also known as wavelet kernels: Haar~\cite{haar1910theorie}, 
CDF~9/7, and CDF~8/4~\cite{cohen1992biorthogonal}.
%
Each wavelet kernel has its advantages and disadvantages in practical use.

EAVL is a scientific visualization framework that aims to better support
extreme-scale analysis and visualization on the many-core architectures.
%
It provides data structure and operators for parallel computing to the 
users, while hiding the implementation details on a certain architecture
(CUDA, OpenMP, OpenCL, etc.).

I plan to implement my project based on both VAPOR and EAVL. 
%
On the one hand, VAPOR has already got DWT implemented in serial, 
including three different wavelet kernels.
%
Starting from VAPOR enables me to focus more on the parallel computing 
component of my project, rather than the details of DWT.
%
On the other hand, EAVL provides a unified programming interface for 
different many-core architectures, which essentially eases the effort
of porting my code from one architecture to another.

My project plan has three phases:
\begin{enumerate}
\item Understand the sequential DWT code from VAPOR, 
and identify the part(s) that can be parallelized.
%
\item Parallelize the sequential DWT code under the EAVL framework.
%
\item Run performance tests and tune my code to achieve a stable version
of DWT on many-core architectures.
\end{enumerate}


\section{Expected Results}
%
The expected result of this project is a stable and practically useful 
implementation of the DWT on many-core architectures.
%
Along this process, I will gain more understanding on parallel
computing, as well as the wavelet transform itself.
%
The ultimate goal is to get myself ready for algorithm designing and 
programming for the many-core architectures in the future.





\bibliographystyle{plain}
\bibliography{main}
\end{document}
